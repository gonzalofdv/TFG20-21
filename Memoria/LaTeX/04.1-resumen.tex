%%ESPAÑOL
\noindent Con el aumento del uso de las tecnologías en todos los aspectos del día a día, sucede a la vez un aumento de las amenazas hacia estas y, entre ellas, una de las principales amenazas es el ransomware, un tipo de malware que bloquea equipos o cifra información impidiendo el acceso a ella y pide un rescate económico. El daño que provocan estos ataques es cada vez mayor, con objetivos como las \gls{SME}, que tienen medios más limitados para protegerse o luchar contra estas amenazas. Es por ello que este estudio arranca con la idea de desarrollar una herramienta de fácil acceso e implantación que sirva de solución contra ataques ransomware.
En este trabajo se presentan tres partes. La primera, la construcción de un laboratorio de análisis de malware para la ejecución y análisis de muestras de ransomware en un entorno seguro y de fácil despliegue. La segunda, a partir de información de 15352 muestras de ransomware y de programas benignos, el desarrollo de dos \textit{dataset} de una extensión considerablemente mayor a los encontrados en otros trabajos relacionados con la detección de malware. Para construir estos \textit{datasets} se ha hecho uso de las llamadas a la \gls{API} de Windows de las muestras analizadas, que reflejan el comportamiento del software y su interacción con el sistema. Después de la limpieza de los datos y la selección de características, el primer \textit{dataset} consta de 714 muestras y el segundo de 6630. La tercera y última parte, el desarrollo de un modelo de inteligencia artificial utilizando diferentes algoritmos de aprendizaje automático capaz de identificar muestras de ransomware, alimentado por los \textit{datasets} mencionados previamente.
Los resultados varían dependiendo del algoritmo utilizado y el \textit{dataset} en el que se aplica, llegando a conseguir una precisión del 98\% y una exactitud del 97\% haciendo uso del algoritmo \gls{RF} sobre el \textit{dataset} de mayor extensión.\\

\vspace{0.2cm}
\noindent \textbf{Palabras clave}: Análisis de Malware, Aprendizaje Automático, Ciberseguridad, Robo de Datos, Sandboxing, Selección de Características, Ransomware.

%\change{ordenar alfab\'eticamente} HECHO
