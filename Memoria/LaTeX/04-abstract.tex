%%INGLES
\noindent Due to the increasing use of technologies in everyday life, there is an increase of threats against them and one of the main threats is ransomware, a malware that blocks the device or encrypts its information, preventing the accesss to it and then asks for an economic ransom. The damage caused by these attacks is increasingly growing, with targets such as \gls{SME}, which have limited resources to deal with these threats. Consequently, this study starts with the idea of developing a simple tool with easy access to detect ransomware attacks.
This work is threefold. The first part involves the construction of a malware analysis laboratory to execute and analyse ransomware samples in a safe and simple to deploy environment. The second part, based on information of 15352 ransomware and benign software samples, deals with the development of two datasets of a considerably larger extension than those found in other studies related to malware detection. To build both datasets we use Windows \gls{API} calls made by the analysed samples, which reflect software behaviour and system interaction. After the cleanup of data and feature selection, the first dataset is made up of 714 samples and the second one 6630. The third and last part involves the development of an artificial intelligence model, feeding the datasets to it and using various machine learning algorithms that are able to identify ransomware among the samples.
The results are different depending on the machine learning algorithm and the datasets used. The best results obtained are 98\% accuracy and 97\% precision using the \gls{RF} algorithm on the largest dataset.

\vspace{0.2cm}
\noindent \textbf{Keywords}: Cybersecurity, Data Theft, Feature Selection, Machine Learning, Malware Analysis, Sandboxing, Windows \gls{API}, Ransomware.


%\change{ordenar alfab\'eticamente} HECHO

